\documentclass[12]{article}

\usepackage{geometry}
\usepackage{amsmath, amsthm, amssymb}
\usepackage{graphicx}
\usepackage{tikz}
\usepackage{booktabs} % See the package documentation for guidelines on formal tables: https://ctan.org/pkg/booktabs
\usepackage{verbatim} % Used to typeset, for example, code snippets or pseudo-code for algorithms.
\usepackage{dsfont} % Extra fontset for helpful mathematics symbols, e.g. \mathds{1}
\usepackage{etoolbox} % Used to allow boolean variables for use in the title page
\usepackage{import}
\usepackage{lipsum}
\usepackage{subcaption}
\usepackage{float}
\usepackage{enumitem}
\usepackage{tabularx}
\usepackage{array}
\usepackage{pdfpages}
\usepackage{mathtools}
\usepackage{hyperref}
\newcolumntype{C}[1]{>{\centering\arraybackslash}m{#1}}
\newcommand{\R}{\mathbb{R}}
\newcommand{\Q}{\mathbb{Q}}
\newcommand{\C}{\mathbb{C}}
\newcommand{\N}{\mathbb{N}}
\newcommand{\Z}{\mathbb{Z}}
\newcommand{\T}{\mathbb{T}}
\newcommand{\cA}{\mathcal{A}}
\newcommand{\cB}{\mathcal{B}}
\newcommand{\cD}{\mathcal{D}}
\newcommand{\cP}{\mathcal{P}}
\newcommand{\cM}{\mathcal{M}}
\newcommand{\abs}[1]{\left\lvert #1 \right\rvert}
\newcommand{\norm}[1]{\left\lVert #1 \right\rVert}
\newcommand{\set}[2]{\left\{#1 \ : \ #2\right\}}
\newcommand{\conv}[1]{\underset{#1}\longrightarrow}
\newcommand{\Mod}[1]{\ (\mathrm{mod}\ #1)}
\newcommand{\Supp}[0]{\ \mathrm{Supp}\ }
\DeclarePairedDelimiter\ceil{\lceil}{\rceil}
\DeclarePairedDelimiter\floor{\lfloor}{\rfloor}
\DeclareMathOperator{\lcm}{lcm}
\newcommand{\Cross}{\mathbin{\tikz [x=1.4ex,y=1.4ex,line width=.2ex] \draw (0,0) -- (1,1) (0,1) -- (1,0);}}

\newcommand\restr[2]{{% we make the whole thing an ordinary symbol
		\left.\kern-\nulldelimiterspace % automatically resize the bar with \right
		#1 % the function
		\vphantom{\big|} % pretend it's a little taller at normal size
		\right|_{#2} % this is the delimiter
}}
% Custom math operators (analogous to \lim, \sup, etc).
\DeclareMathOperator{\id}{id}
\DeclareMathOperator{\subspan}{span}
\DeclareMathOperator{\sgn}{sgn}
\DeclareMathOperator{\diam}{Diam}

\newtheorem{thm}{Theorem}[section] % Numbering is impacted by [chapter]; could do [section] or [subsection] also.
\newtheorem{lem}{Lemma} % The [thm] argument says to number Lemma in sequence with Theorem.
\newtheorem{prop}[thm]{Proposition}
\newtheorem{cor}[thm]{Corollary}
\newtheorem{conj}[thm]{Conjecture}
\newtheorem{question}{Question}
% These environments are unnumbered and will not count toward the numbering.
%\newtheorem*{question}{Question}
\newtheorem*{answer}{Answer}
\newtheorem*{conjecture}{Conjecture}
\newtheorem*{claim}{Claim}
% These environments are definitions; they have a different style (bold label, standard font).
\theoremstyle{definition}
\newtheorem{defn}[thm]{Definition} % These definitions are also numbered in sequence with Theorem.
\newtheorem{eg}{Example}
\newtheorem{rem}[thm]{Remark}
\newtheorem{obs}{Observation}

\title{ \vspace{-3cm} Ph.D. Working Notes }
%\author{Tao Gaede}

\begin{document}
	\maketitle
	
	\section{Crescent Labelled Trees}
	
	Let $T$ be a tree of order $n$.  A crescent labelling of $T$ is a map $L: E(T) \mapsto \{1,2, \ldots, t\}$, such that the distance multiset of $L(T)$ is of the form $\{d_1^1,d_2^2, \ldots, d_{n-1}^{n-1}\}$.  The diameter of $T$, denoted $\diam(T)$, is the length of the $(u,v)$-path in $T$.  The max degree of $T$ is denoted $\Delta(T)$.  %We use $\ell$ to denote the number of leaves of $T$.
	

\begin{lem}[Basic Diameter Lower Bound]
	Let $t$ be a positive integer.  If $L(T)$ is a crescent labelling of the tree $T$ with weights $\{1,2,\ldots,t\}$, then $\diam(T) \geq \tfrac{n-1}{t}$.
\end{lem}

\begin{proof}
	Since there are at least $n-1$ distinct distances, there is a distance $d$ with value at least $n-1$.  Let $u,v \in V(T)$ such that $d(u,v) = d$, then since $t$ is the max edge weight, this means that the number of edges on a $(u,v)$-path is at least $\tfrac{d}{t} \geq \tfrac{n-1}{t}$.
\end{proof}
	
	
	For a pair of vertices $u,v \in V(T)$, we denote the $(u,v)$-path in $T$ as $P(u,v)$.  Lemma \ref{Lemma-DegreeClasses} below generalizes the observation underlying the maximum degree upper bound of $\sim$ $\sqrt{2n}$.
	
	\begin{lem}\label{Lemma-DegreeClasses}
		Let $T$ be a tree of order $n$.  For every $i \in [1,n-1]$, $M \in V(T)$, and $j \in \mathcal{N}(M)$, define
		$$D_j := \{u \in V(T) \setminus \{M\}: d(u,M) = d_i, j \in P(u,M)\}.$$
		Then distance $2d_i$ occurs with multiplicity at least $\sum_{j < k}^{\deg(M)} |D_j|\cdot |D_k|$.
	\end{lem}

	\begin{proof}
		Let $M \in V(T)$ and $i \in [1,n-1]$.  Since $T$ is a tree, there is always a unique $(u,v)$-path for all $u,v \in V(T)$.  So, for each $u \in D_j$ and $v \in D_k$, the $(u,v)$-path must go through $M$, which means $d(u,v) = d(u,M) + d(M,v) = 2d_i$.  There are $|D_j| \cdot |D_k|$ such $u$ and $v$ pairs, so indeed $2d_i$ has multiplicity at least $\sum_{j < k}^{\deg(M)} |D_j|\cdot |D_k|$. \qedhere
	\end{proof}
	
	Now we apply the lemma to get a condition on crescent labelled trees.
	
	\begin{prop}[Max Multiplicity Condition]
		Let $L(T)$ be a crescent labelling of a tree $T$.  Then for every $i \in [1,n-1]$, $M \in V(T)$, and $j \in \mathcal{N}(M)$, 
		$$\sum_{j < k}^{\deg(M)} |D_j|\cdot |D_k| < n.$$
	\end{prop}
	
	\begin{proof}
		Since $L(T)$ is a crescent labelling of $T$, no distance can have multiplicity greater than $n-1$and $T$ is a tree.  Since $T$ is a tree, it follows by Lemma \ref{Lemma-DegreeClasses} that for each vertex $M \in V(T)$, $i \in [1,n-1]$, $\sum_{j < k}^{\deg(M)} |D_j|\cdot |D_k| < n$. \qedhere
	\end{proof}
	
	Next is a general lemma on $\{1,2,\ldots,t\}$-words containing subwords with $t-1$ consecutive $1s$.
	
	\begin{lem}[Arithmetic Condition]\label{Lemma-ArithmeticCondition}
		Let $k \geq 2t$.  Let $\mathbf{w}$ be a $\{1,2,\ldots, t\}$-word with length $k$.  If $w_{a-t+2} = w_{a-t+3} = \cdots = w_{a} = 1$ for some $a \in \{1,2, \ldots, k\}$, then each value 
		$$1,2, \ldots, \max\biggr\{\sum_{i = 1}^{a}w_i, \sum_{i=a-t+2}^{k}w_i\biggr\}$$ 
		occurs as a partial sum in $\mathbf{w}$.
	\end{lem}
	
	\begin{proof}
		Suppose without loss of generality that $\sum_{i = 1}^{a}w_i \leq \sum_{i=a-t+2}^{k}w_i$.  Then it is sufficient to show that every value $1,2,\ldots, \sum_{i=a-t+2}^{k}w_i$ occurs as a partial sum in $\mathbf{w}$.  Call $w_{a-t+2},w_{a-t+3}, \ldots, w_a$ the \emph{unit segment} of $\mathbf{w}$ and $w_{a+1},w_{a+2},\ldots,w_k$ the \emph{non-unit segment} of $\mathbf{w}$.  We proceed by induction on the number of terms $r$ in the non-unit segment of $\mathbf{w}$.  When $r = 1$, $w_{a+r} \in \{1,\ldots,t\}$, and since the unit segment has $t-1$ 1s, for each $j \in \{1,2,\ldots,t-1\}$, we have the partial sums $j = \sum_{i=0}^{j-1}w_{a-i}$.  Then the values between $w_{a+r}$ and $\sum_{i=a-t+2}^{a+r}w_i$ are of the form $w_{a+r}+ \sum_{i=0}^{j-1}w_{a-i}$.  For the inductive step, the values $1,2,\ldots,\sum_{i=a-t+2}^{a+r-1}w_i$ occur at least once by inductive hypothesis.  We have that $w_{a+r} \in \{1,2,\ldots,t\}$ and the values between $\sum_{i=a+1}^{a+r-1}w_i$ and $\sum_{i=a+1}^{a+r}w_i$ can be obtained from $\sum_{i=a+1-j}^{a+r-1}w_i$ for each $j \in \{1,2,\ldots,t-1\}$.  Then similarly the values between $\sum_{i=a+1}^{a+r}w_i$ and $\sum_{i=a-t+2}^{a+r}w_i$ are $\sum_{i=a+1-j}^{a+r}w_i$ for $j \in \{1,2,\ldots,t-1\}$. \qedhere
	\end{proof}
	
	We now apply this arithmetic lemma to crescent labelled trees to show that when there are many consecutive 1s on a path, the path cannot be too long with many large weight edges.
	
	\begin{prop}
		Let $L(T)$ be a crescent labelling of a tree $T$ with edge weights in $\{1,2,\ldots,t\}$.  Then for every path $P = (v_1v_2,v_2v_3,\ldots,v_{t-1}v_t)$ in $T$ such that $w(v_iv_{i+1}) = 1$ for $i \in \{1,2,\ldots,t-1\}$, it follows that $\max\{d(v_1,u): u \in V(T)\} < n$ and $\max\{d(v_t,u): u \in V(T)\} < n$.
	\end{prop}
	
	\begin{proof}
		Let $T$ be a tree with a path $P$ specified in the proposition statement and $L(T)$ a crescent labelling.  It is sufficient to show that $\max\{d(v_1,u): u \in V(T)\} < n$ since the case for $v_t$ is similar.  Let $u' \in V(T)$ such that $d(v_1,u') = \max\{d(v_1,u): u \in V(T)\}$.  By Lemma \ref{Lemma-ArithmeticCondition}, every distance $1,2,\ldots, d(v_1,u')$ occurs at least once.  Since $L(T)$ is a crescent labelling, there can be at most $n-1$ distinct distances, so $d(v_1,u') < n$ as desired. \qedhere
	\end{proof}
	The implication for when $t=2$ is quite strong since this imposes a max distance condition on vertices incident to edges with weight $1$.
	\begin{cor}
		Let $L(T)$ be a crescent labelling of a tree $T$.  If $t = 2$, then every vertex incident to an edge with weight $1$ has max distance at most $n-1$.
	\end{cor}
	
	%\begin{prop}
	%	Let $L(T)$ be a crescent labelling of a tree $T$.  For every path $P$ of length $t-1$ whose edges all have weight $1$, each end vertex $v \in P$ satisfies 
	%	$$\max\{d(v,u):u \in V(T)\} \leq n.$$
	%\end{prop}
	
	What follows is a basic lemma about trees that may turn out to be useful in case parameterizing by number of leaves becomes sensible.
	
	\begin{lem}[From Chartrand and Lesniak's text ``Graphs and Digraphs" 4th edition]
		Let $T$ be a tree with $n_i$ vertices with degree $i$, where $i \in \{1,2,\ldots,\Delta(T)\}$.  Then $n_1 = n_3 + 2n_4 + 3n_5 + \cdots + (\Delta(T)-2)n_{\Delta(T)} + 2$.
	\end{lem}
	
	\begin{proof}
		Note that $n = \sum_{i=1}^{\Delta(T)}n_i$.  Since $T$ is a tree, 
		$$\sum_{i=1}^{\Delta(T)} in_i  = \sum_{v \in V(T)} \deg(v) = 2(n-1) = 2\biggr(\sum_{i=1}^{\Delta(T)}n_i \biggr) - 2.$$  
		Rearranging gives $2+ \sum_{i=1}^{\Delta(T)} (i-2)n_i = 0.$
	\end{proof}
	
	\begin{cor}
		If $T$ is a tree, then $\sum_{i=3}^{\Delta(T)}(i-2)n_i < n_1$
	\end{cor}
	
\end{document}