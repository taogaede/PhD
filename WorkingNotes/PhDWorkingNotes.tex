\documentclass[12]{article}

\usepackage{geometry}
\usepackage{amsmath, amsthm, amssymb}
\usepackage{graphicx}
\usepackage{tikz}
\usepackage{booktabs} % See the package documentation for guidelines on formal tables: https://ctan.org/pkg/booktabs
\usepackage{verbatim} % Used to typeset, for example, code snippets or pseudo-code for algorithms.
\usepackage{dsfont} % Extra fontset for helpful mathematics symbols, e.g. \mathds{1}
\usepackage{etoolbox} % Used to allow boolean variables for use in the title page
\usepackage{import}
\usepackage{lipsum}
\usepackage{subcaption}
\usepackage{float}
\usepackage{enumitem}
\usepackage{tabularx}
\usepackage{array}
\usepackage{pdfpages}
\usepackage{mathtools}
\usepackage{hyperref}
\newcolumntype{C}[1]{>{\centering\arraybackslash}m{#1}}
\newcommand{\R}{\mathbb{R}}
\newcommand{\Q}{\mathbb{Q}}
\newcommand{\C}{\mathbb{C}}
\newcommand{\N}{\mathbb{N}}
\newcommand{\Z}{\mathbb{Z}}
\newcommand{\T}{\mathbb{T}}
\newcommand{\cA}{\mathcal{A}}
\newcommand{\cB}{\mathcal{B}}
\newcommand{\cD}{\mathcal{D}}
\newcommand{\cP}{\mathcal{P}}
\newcommand{\cM}{\mathcal{M}}
\newcommand{\abs}[1]{\left\lvert #1 \right\rvert}
\newcommand{\norm}[1]{\left\lVert #1 \right\rVert}
\newcommand{\set}[2]{\left\{#1 \ : \ #2\right\}}
\newcommand{\conv}[1]{\underset{#1}\longrightarrow}
\newcommand{\Mod}[1]{\ (\mathrm{mod}\ #1)}
\newcommand{\Supp}[0]{\ \mathrm{Supp}\ }
\DeclarePairedDelimiter\ceil{\lceil}{\rceil}
\DeclarePairedDelimiter\floor{\lfloor}{\rfloor}
\DeclareMathOperator{\lcm}{lcm}
\newcommand{\Cross}{\mathbin{\tikz [x=1.4ex,y=1.4ex,line width=.2ex] \draw (0,0) -- (1,1) (0,1) -- (1,0);}}

\newcommand\restr[2]{{% we make the whole thing an ordinary symbol
		\left.\kern-\nulldelimiterspace % automatically resize the bar with \right
		#1 % the function
		\vphantom{\big|} % pretend it's a little taller at normal size
		\right|_{#2} % this is the delimiter
}}
% Custom math operators (analogous to \lim, \sup, etc).
\DeclareMathOperator{\id}{id}
\DeclareMathOperator{\subspan}{span}
\DeclareMathOperator{\sgn}{sgn}
\DeclareMathOperator{\diam}{Diam}

\newtheorem{thm}{Theorem}[section] % Numbering is impacted by [chapter]; could do [section] or [subsection] also.
\newtheorem{lem}{Lemma} % The [thm] argument says to number Lemma in sequence with Theorem.
\newtheorem{prop}[thm]{Proposition}
\newtheorem{cor}[thm]{Corollary}
\newtheorem{conj}[thm]{Conjecture}
\newtheorem{question}{Question}
% These environments are unnumbered and will not count toward the numbering.
%\newtheorem*{question}{Question}
\newtheorem*{answer}{Answer}
\newtheorem*{conjecture}{Conjecture}
\newtheorem*{claim}{Claim}
% These environments are definitions; they have a different style (bold label, standard font).
\theoremstyle{definition}
\newtheorem{defn}[thm]{Definition} % These definitions are also numbered in sequence with Theorem.
\newtheorem{eg}{Example}
\newtheorem{rem}[thm]{Remark}
\newtheorem{obs}{Observation}

\title{ \vspace{-3cm} Ph.D. Working Notes }
%\author{Tao Gaede}

\begin{document}
	\maketitle
	
	\section{Trees}
	
	Let $T$ be a tree of order $n$.  A crescent labelling of $T$ is a map $L: E(T) \mapsto \{1,2, \ldots, t\}$, such that the distance multiset of $L(T)$ is of the form $\{d_1^1,d_2^2, \ldots, d_{n-1}^{n-1}\}$.  The diameter of $T$, denoted $\diam(T)$, is the length of the maximum shortest $(u,v)$-path in $T$.  The max degree of $T$ is denoted $\Delta(T)$.  %We use $\ell$ to denote the number of leaves of $T$.
	
	\subsection{Path Perspective: Crescent Labellings and $r$-Bounding Paths}
	
	%When $t = 2$, crescent labellings are common in the following two cases: (1) when $\diam(T)$ is large with $\Delta(T)$ and $\ell$ small; and (2) when $\diam(T)$ is small with $\Delta(T)$ and $\ell$ large.  In this section, we show that for $n$ sufficiently large, crescent labellings do not work in the remaining cases.
	
	%First we need an important lemma.
	
	This section establishes a simple arithmetic condition on long paths in a crescent labelled tree.
	
	\begin{lem}[Arithmetic Condition]\label{Lemma-ArithmeticCondition}
		Let $k \geq 2t$.  Let $\mathbf{w}$ be a $\{1,2,\ldots, t\}$-word with length $k$ containing at least two disjoint subwords of $t-1$ 1s.  Let $w_{a-t+2} = w_{a-t+3} = \cdots = w_{a} = 1 = w_b = w_{b+1} = \cdots = w_{b+t-2}$, where $a$ and $b$ are the smallest and largest satisfying $a<b$, respectively.  Then all distances $1,2, \ldots, \sum_{i = a}^{b-1}w_i$ occur with multiplicity at least $2$.
	\end{lem}

	\begin{proof}
		Let $m = \sum_{i = a}^{b-1}w_i$ and observe that because $w_a = w_b = 1$, $m = \sum_{i = a+1}^{b}w_i$ as well.  It is sufficient to show that for any such $\mathbf{w}$, the following set inclusion holds: 
		$$\{1,2,\ldots, m\} \subseteq \{d(w_{a+1-j},w_i): j \in [0,t-1], i \in [a-j+2,b-1]\}.$$
		
		This is because, if true, the same property will hold for the distances between the all $1$s subwords involving $w_{b}$, $w_{b+1}$, $\ldots$, $w_{b+t-2}$, which means that each such distance occurs at least twice.  We show this by induction on the length of the word $\mathbf{w}' = (w_{a-t+2},w_{a-t+3}, \ldots, w_{a}, w_{a+1}, \ldots, w_{a+r-1})$, which we denote by $r+t-1$.  The statement clearly holds for $r=0$.  If $r=1$, then $\mathbf{w}' = (1^{t-1},1)$ or $\mathbf{w}' = (1^{t-1},u)$ for some $u \in [2,t]$, where concatenation means repeated consecutive word elements.  In the former case, distances $1,2, \ldots, t$ clearly occur at least once, and in the latter, distances $1,2, \ldots, t-1+u$ each occur as well since for any $v \in [0,u-1]$, $t-1+u-v$ is the sum of all but the first $v$ elements of $\mathbf{w}'$.  Suppose the inductive hypothesis holds for some $r \geq 1$ and set $m' = \sum_{i=a-t+2}^{a+r-1}w_i$.  That is, we assume that the $\{1,2, \ldots, t\}$-word $\mathbf{w}' = (w_{a-t+2}, w_{a-t+3}, \ldots, w_a,w_{a+1}, \ldots, w_{a+r-1})$ attains the distances $1,2,\ldots, m'$.  Then the length $(r+1)+t-1$ word $\mathbf{w}'$ with a $1$ appended to the end clearly attains all distances $1,2,\ldots, m'+1$.  The other case is when $u$ is appended to the end of $\mathbf{w}'$ where $u \in [2,t]$.  In this case, the word attains the distance $\sum_{i=a-t+2+v}^{a+r}w_i = m'+v$ for each $v \in [2,u]$, which is what we wanted to show.
		
		Thus in particular, the distances $1,2, \ldots, m$ each occur at least once in 
		$$\{d(w_{a+1-j},w_i): j \in [0,t-1], i \in [a-j+2,b-1]\}$$
		%$$\{d(w_a,w_i):i \in [a+1,b-1]\} \cup \{d(w_{a+1},w_i): i \in [a+2,b-1]\}$$ 
		as well as at least once in 
		$$\{d(w_{b-1+j},w_i): j \in [0,t-1], i \in [a+1,b+j-2]\},$$
		%$$\{d(w_b,w_i):i \in [a+1,b-1]\} \cup \{d(w_{b-1},w_i): i \in [a+1,b-2]\}.$$ 
		as desired. \qedhere
	\end{proof}

\begin{lem}[Basic Diameter Lower Bound]
	Let $t$ be a positive integer.  If $L(T)$ is a crescent labelling of the tree $T$ with weights $\{1,2,\ldots,t\}$, then $\diam(T) \geq \tfrac{n-1}{t}$.
\end{lem}

\begin{proof}
	Since there are at least $n-1$ distinct distances, there is a distance $d$ with value at least $n-1$.  Let $u,v \in V(T)$ such that $d(u,v) = d$, then since $t$ is the max edge weight, this means that the number of edges on a $(u,v)$-path is at least $\tfrac{d}{t} \geq \tfrac{n-1}{t}$.
\end{proof}

\subsubsection{$t$-Bounded Paths}

Let $L(T)$ be a labelling of $T$ with edge weights from $\{1,2,\ldots, t\}$.  If a path $P \subseteq T$ is labelled by $L$ in such a way that \textbf{(1)} all edge weights are in $\{1,2,\ldots, r\}$ for some $1 \leq r \leq t$, and \textbf{(2)} it is appended on each side by $r-1$ consecutive edges with weight $1$, then we call $P$ an \emph{$r$-bounded path} of $T$.  Define $b_L(T,r)$ to be the maximum weight sum over all edges of a $r$-bounded path of $T$ under $L$.

\begin{prop}
	Let $L(T)$ be a crescent labelling of $T$ with weights $\{1,2,\ldots,t\}$.  Suppose $L(T)$ contains a $r$-bounded path $P = (e_1,e_2,\ldots,e_k)$ of $T$.  Then every distance $1,2,\ldots, b_L(T,r)$ occurs with multiplicity at least $2$. %path $P = (p_1p_2,p_2p_3,\ldots,p_{k-1}p_k)$ such that both $w(p_1p_2) = w(p_2p_3) = \cdots = w(p_{t-1}p_t) = 1,$	and	$w(p_{k-t-2}p_{k-t-1}) = w(p_{k-t-1}p_{k-t}) = \cdots = w(p_{k-1}p_k) = 1,$ and set $m:= k-2(t-1)$.  Then every distance $1,2,\ldots, m$ occurs with multiplicity at least $2$.
\end{prop}
\begin{proof}
	The statement follows immediately from Lemma \ref{Lemma-ArithmeticCondition}.
\end{proof}

\begin{prop}
	Let $L(T)$ be a crescent labelling of $T$ with weights $\{1,2,\ldots,t\}$.  Then $b_L(T,r) \leq n-1$ for every $r \in \{1,2,\ldots,t\}$.
\end{prop}

\begin{proof}
	Suppose $b_L(T,r) \geq n-1$.  Then by the proposition, every distance $1,2,\ldots,n-1$ has multiplicity at least $2$, which contradicts $L(T)$ inducing $n-1$ distinct distances such that exactly one is unique.
\end{proof}

\begin{cor}
	Let $L(T)$ be a crescent labelling of $T$ with weights $\{1,2,\ldots,t\}$.  Define $a_i$ to be the number of weight $i$ edges on an $r$-bounding path of $T$ with length $k$ and maximum weight sum under $L$.  Then $\sum_{i=1}^r ia_i \leq n-1$ and $\sum_{i=1}^r a_i = k$.
\end{cor}

\begin{prop}
	Let $L(T)$ be a crescent labelling of $T$ with weights $\{1,2,\ldots,t\}$.  If $P = (e_1$, $e_2$, $\ldots$, $e_k)$ is an $r$-bounding path of $T$ under $L$, then $k \leq n-1 - \sum_{i=2}^r (i-1)a_i$.
\end{prop}

The next proposition suggests that, when $t =2$, the larger the maximum distance becomes (over $n-1$), the more weight $2$ edges there should be on the tails of the path inducing maximum distance.

\begin{prop}
	Let $L(T)$ be a crescent labelling with weights $\{1,2,\ldots, t\}$.  For every path $ F= (v_1v_2,v_2v_3,\ldots,v_{t-1}v_t)$ satisfying $w(v_iv_{i+1}) =1$ for each $i \in [1,t-1]$, define $X$ and $Y$ to be the vertices of $T$ connected to $v_1$ and $v_t$, respectively, but not through those in $F$.  Then for every $x \in X$, $d(v_1,x) \leq n$ and for every $y \in Y$, $d(v_t,y) \leq n$.
\end{prop}

\begin{proof}
	Let $P = (e_1,e_2,\ldots,e_k)$ be a path such that the longest distance in $L(T)$ is $\sum_{i=1}^k w(e_i)$.  Let $P' =  (e_a,e_{a+1}, \ldots, e_{b})$ be any smallest path satisfying $\sum_{i = a}^b w(e_i) \geq n$ for some $1 \leq a < b \leq k$.  If there is a $j \in \{1,2,\ldots, a-1\} \cup \{b+1,b+2,\ldots,k\}$ such that $w(e_j) = w(e_{j+1}) = \cdots = w(e_{j+t-1}) =1$, then every distance $1,2,\ldots,n$ occurs at least once (by similar reasoning to the proof of Lemma \ref{Lemma-ArithmeticCondition}, a contradiction.  So, if $t = 2$, then all edges connected to $P \setminus P'$ have to have weight $2$ since otherwise there would be a path beginning with a weight $1$ edge containing $P'$, which again means there would be more than $n-1$ distinct distances.  In general, there can be no path of $t-1$ consecutive weight $1$ edges in $P \setminus P'$. \qedhere
\end{proof}

\begin{prop}
	Let $L(T)$ be a crescent labelling with both weights $\{1,2\}$.  Let $P$ be a path in $T$ where $P= (e_1,e_2,\ldots,e_k)$ such that the following two properties hold:
	\begin{enumerate}
		\item $\sum_{i=1}^{k} w(e_i) = d_{max}$.
		
		\item $P$ contains a subpath $P' = (e_a,e_{a+1},\ldots,e_{a+b-1})$ satisfying $\sum_{i=a}^{a+b-1} w(e_i) \in [n+1, n+2]$ where $a$ and $b$ are chosen so that the maximum number of weight $2$ edges are to the ``right" of $b$.
		
	\end{enumerate}
	  Then $k \leq \tfrac{3n}{4}$.
\end{prop}

\begin{proof}
	Note by proposition, no edge $e_h$ for any $h \in [1,a-1] \cup [a+b,k]$ satisfies $w(e_h) = 1$.  Suppose for a contradiction that $k> 3n/4$.  Then either $P'$ contains an edge with weight $1$ or it does not.  If $P'$ contains only weight $2$ edges, then since $k \geq 3n/4$ and there exists an edge with weight $1$, Suppose for some $j \in [a,a+b-1]$, $w(e_{j}) = 1$.    Thus the sums $\sum_{i=1}^j w(e_i) \leq n$ and $\sum_{i=j}^{k} w(e_i) \leq n$ both hold.  But since $\sum_{i=a}^{a+b-1} w(e_i) \in [n+1, n+t]$, we have that, say, $n+1 = a_1 + 2a_2$, where $a_1$ and $a_2$ are the number of $1$s and $2$s in $P'$.  Since $b-a > n/2$ and $k > 3n/4$, either $w(e_1) = \cdots = w(e_{n/4}) =2 $ or $w(e_k) = w(e_{k-1}) = \cdots = w(e_{k-1-n/4}) = 2$.  Since $w(e_j) = 1$ and $j \in [a,a+b-1]$, each of these tail edges with weight $2$ correspond to at least $2$ distinct distances each, implying that there are at least $n/2$ distinct distances from these tails.  Then for each $h \in [1,j]$ there is a distinct distance as well.  That is, if $j = n/C$, and if $J_1$ and $J_2$ are the number of weights $1$ and $2$ edges with index larger than $j$, respectively, then there are $2J_2 + J_1$ distinct distances to the right of $j$ and $n/C$ to the left.  Thus we require $n-1 \geq 2J_2 + J_1 + n/C$.  If there are $n/4$ twos and $n/4$ ones and $C > n/4$, then we get a contradiction.  Note that we require $J_2 + J_1 + n/C = k$ and we know for sure that $J_2 > n/4$.
\end{proof}

\begin{lem}
	Let $L(T)$ be a crescent labelling of $T$ using weights $\{1,2,\ldots,t\}$.  Then for every $r \in \{2,3,\ldots,t\}$, $L(T)$ cannot contain a subpath $P = (e_1,e_2,\ldots,e_{(n-1)/r})$ such that $w(e_i) \in \{1,2,\ldots,r\}$ where $i \in \{1,2,\ldots,(n-1)/r\}$ and $P$ is preceded by a path of $r-1$ weight $1$ edges.
\end{lem}

%\begin{lem}
%	Let $P = (e_1,e_2,\ldots,e_k)$ be a path in $L(T)$.  Define $a(i,x) := |\{e_j: w(e_j) = i, j \leq x\}|$.  If $k > \tfrac{n-1}{2} + 2\sqrt{n}$ and $a(1,\tfrac{n-1}{2}) \geq \sqrt{n}$, then the distances $1,2,\ldots, \sqrt{n}$ each occur with multiplicity at least $\sqrt{n}$.
%\end{lem}

	\iffalse
	\begin{lem}
		Let $k$ be a positive integer satisfying $k \geq 5$.  If an edge-weighted path $P$ with length $k$ has at least three edges with weight $1$, then all distances $1,2,\ldots, k-2$ occur with multiplicity at least $2$.
	\end{lem}

\begin{proof}
	Let $\mathbf{w}$ be a $\{1,2\}$ word with length $k$.  By assumption we know that at least three of the elements of $\mathbf{w}$ are $1$s.  Suppose $\mathbf{w}_a = \mathbf{w}_b = \mathbf{w}_c = 1$ where $1 \leq a < b < c \leq k$.  First observe that since $k \geq 5$ and there are at least three $1$s, distances $1$, $2$, $3$, and $4$ each occur at least twice as a distance.  To show that the remaining distances up to $k-2$ each occur at least twice, we argue by induction in regards to the partial sums with the elements from near $\mathbf{w}_a$ and those to the right.  Suppose without loss of generality that $a < k/2$; note that if for all $i < k/2$, $\mathbf{w}_i = 2$, ten just flip $\mathbf{w}$ so that $\mathbf{w}_i = \mathbf{w}_j$ for all $1 \leq j \leq k$.  
	
	Note that $2122$ contains two $5$s.  $21221$, $21222$, $212221$, $212222$, $2122211$, $2122212$, $21222121$, $21222122$
	
	
	Suppose the partial sums in the subword $\mathbf{w}^{(m)} = (\mathbf{w}_a, \mathbf{w}_{a+1}, \ldots, \mathbf{w}_{a+m-1})$ of the form
	$$\sum_{i = a}^{}$$
	%contains all distances $1,2, \ldots, m-1$.  If $\mathbf{w}_{a+m} = 2$, then $\sum_{\mathbf{w}_a, \mathbf{w}_{a+1}, \ldots, \mathbf{w}_{a+m-1})$
\end{proof}
\fi

\iffalse
	We suppose going forward that the diameter path $D$ of $T$ is labelled in a way so that there are at least three edges with weight $1$.
	
	\begin{lem}
		If $\ell > \diam(T)/2 + 1$, then the maximum distance occurs uniquely in $T$ on the diameter path of $L(T)$.
	\end{lem}
	
	Our argument is in four parts:
	
	\begin{lem}[high $\Delta(T)$]
		If $\Delta(T) \geq \frac{t(\sqrt{8n+1}+1)}{2}$, then no crescent labelling exists.
	\end{lem}
	
	\begin{proof}
		We suppose that $\Delta(T) \geq \frac{t(\sqrt{8n+1}+1)}{2}$.  Then by PHP, the distances $1+1$ or $2+2$ occurs at least ${\frac{(\sqrt{8n+1}+1)}{2} \choose 2}$ times, and if this number is at least $n$, which it is, then no crescent labelling exists because then some distance occurs with multiplicity greater than $n-1$, a contradiction. \qedhere
	\end{proof}
	
	
	\paragraph{\textbf{A:}	High $\mathbf{\diam(T)}$, high $\mathbf{\ell}$}
	
	Observe that since the maximum distance occurs on the diameter path of $T$, and this distance must be at least $n-1$, it holds that $\diam(T) \geq (n-1)/2 + 3$ (in extreme case where all but three edges on the path are $2$s).
	
	\begin{lem}
		If $\ell > \diam(T)/2 + 1$ and $\Delta(T) < \frac{t(\sqrt{8n+1}+1)}{2}$, then 
	\end{lem}

	\begin{proof}
		There are too many two edge and three edge paths producing distances 3 and 4.  Because the diameter is so big and the number of leaves is so high, there need to be a lot of vertices with moderately high degree -- around $\sqrt{n}$.  In particular, there are at least $n/4$ leaves and 
	\end{proof}
	
	\paragraph{\textbf{B:}	low $\mathbf{\diam(T)}$, low $\mathbf{\ell}$}

	
	\paragraph{\textbf{D:}	low $\mathbf{\diam(T)}$, low $\mathbf{\Delta(T)}$}
\fi
	
\end{document}